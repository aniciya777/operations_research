\documentclass[12pt]{article}
    
\usepackage[english,russian]{babel}

\usepackage[left=2cm,right=1cm,
    top=1cm,bottom=2cm,bindingoffset=0cm]{geometry}
    
\author{Остапчук А.В.}
\title{Задача распределения ресурсов}

\begin{document}
\maketitle
\renewcommand{\abstractname}{Вариант 5}
\begin{abstract}

Для развития двух предприятий П1 и П2 на 4 года выделено $ x $ средств. Количество средств $ y $, вложенных в П1, обеспечивает годовой доход в размере $ 1{,}4 y^{2} $ и уменьшается за год до величины $ 0{,}55 y $. Количество средств $ x - y $, вложенных в П2 обеспечивает годовой доход в размере $ 2 \left(x - y\right)^{2} $ и уменьшается за год до величины $ 0{,}6 x - 0{,}6 y $. Необходимо так распределить выделенные средства по годам планируемого периода, чтобы получить максимальный доход.

\end{abstract}

\noindent\rule{\textwidth}{1pt}\newline

\textit{Решение.}



Период времени продолжительностью 4 лет разобьем на 4 этапов, поставив в соответствие каждому году один этап, т.е. $ N $ = 4, $ k $ = 1, 2, 3, 4. Хотя рассматривается непрерывный процесс, величины $ x $ и $ y $ для наглядности на каждом этапе будем отмечать индексами.

1. Отыскание оптимального решения начинаем с 4-го этапа, в начале которого необходимо распределить средства $ x_3 $, оставшиеся после 3-го этапа. Для этого следует определить оптимальное значение $ y_4 $. Составим выражения для функций, входящих в уравнение:

$$ t_4 ( x_3, y_4 ) = \varphi ( y_4 ) + \xi ( x_3 - y_4 ) = 1{,}4 y_{4}^{2} + 2 \left(x_{3} - y_{4}\right)^{2} ;$$

$$ f_4 ( x_3 ) = \max_{ 0 \le y_4 \le x_3 }{\left[ 1{,}4 y_{4}^{2} + 2 \left(x_{3} - y_{4}\right)^{2} \right]} .$$

Для определения значения переменной $ y_{4} $ на отрезке [0, $ x_{3} $], в которой функция $ t_4 ( x_3, y_4 ) = 1{,}4 y_{4}^{2} + 2 \left(x_{3} - y_{4}\right)^{2} $ принимает наибольшее значение можно использовать метод дифференциального исчисления. Однако, если учесть, что $ x_{3} $ для 4-го этапа есть величина постоянная, нетрудно заметить, что $ g_4 ( x_3, y_4 ) = 1{,}4 y_{4}^{2} + 2 \left(x_{3} - y_{4}\right)^{2} $ – уравнение параболы, ветви которой направлены вверх. Стало быть, наибольшее значение на отрезке [0,~$ x_{3} $] функция принимает на одном из его концов.

Определим значение функции на концах отрезка [0,~$ x_{3} $]:

$$ t_4 ( x_3, y_4 ) = 2 x_{3}^{2} \textrm{ при } y_{4} = 0,$$

$$ t_4 ( x_3, y_4 ) = 1{,}4 x_{3}^{2} \textrm{ при } y_{4} = x_{3}.$$

Так как $ 2 x_{3}^{2} $ > $ 1{,}4 x_{3}^{2} $, то функция $ t_4 ( x_3, y_4 ) $ принимает максимальное значение на отрезке [0,~$ x_{3} $] при $ y_{4} $ = 0, следовательно, $ f_4(x_{3}) = 2 x_{3}^{2} $.

Таким образом, максимальный доход на последнем этапе достигается в том случае, если в начале его все оставшиеся средства вложить в развитие предприятия П2.

Последовательно определим оптимальные распределения средств на 3, 2 и 1-м этапах.

2. Ищем оптимальное распределение для 2 последних этапов – 3-го и 4-го. Средства, доступные после 2-го этапа равны $ x_{2} $. Функциональное уравнение имеет вид:

$$ f_3 ( x_{2} ) = max_{ 0 \le y_3 \le x_{2} }{\left\{ t_3 ( x_{2}, y_3 ) + f_4 ( x_3 ) )\right\}}= max_{ 0 \le y_3 \le x_{2} }{\left\{ 1{,}4 y_{3}^{2} + 2 \left(x_{2} - y_{3}\right)^{2} + 2 x_{3}^{2} \right\}}.$$

Здесь $ x_3 $ – сумма оставшихся средств после 3-го этапа (на 3-м этапе было израсходовано $ y_{3} $ средств на предприятии П1 и $ x_{2} - y_{3} $ – на предприятии П2), т.е.

$$ x_3 = 0{,}55 y_{3} + 0{,}6 x_{2} - 0{,}6 y_{3} = 0{,}6 x_{2} - 0{,}05 y_{3} .$$

Заменяя $ x_3 $ его выражением через $ x_{2} $ и $ y_{3} $, окончательно получаем функциональное уравнение:

$$ f_3 ( x_{2} ) = max_{ 0 \le y_3 \le x_{2} }{\left\{ 2{,}72 x_{2}^{2} - 4{,}12 x_{2} y_{3} + 3{,}405 y_{3}^{2} \right\}}.$$

Находим значение $ y_{3} $, при котором функция, заключенная в фигурные скобки, на отрезке [0,~$ x_{2} $] достигает наибольшего значения (для простоты обозначим ее через $ Z_3 $). Так как $ x_{2} $ – для 3-го этапа есть величина постоянная, то

$$ Z_3 = Z_3 ( y_3 ) =  2{,}72 x_{2}^{2} - 4{,}12 x_{2} y_{3} + 3{,}405 y_{3}^{2} $$

есть уравнение параболы, ветви которой направлены вверх. Наибольшее значение на отрезке [0, $ x_{2} $] функция $ Z_3 ( y_3 ) $ принимает на одном из его концов. Имеем:

$$ Z_3 ( 0 ) = 2{,}72 x_{2}^{2}, $$

$$ Z_3 ( x_{2} ) = 2{,}005 x_{2}^{2}. $$

Поэтому $ f_3 ( x_{2} ) = 2{,}72 x_{2}^{2} $. Следовательно, максимальный доход на 3-м этапе будет достигнут в том случае, если в начале его все оставшиеся средства вложить в развитие предприятия П2.

3. Ищем оптимальное распределение для 3 последних этапов – 2-го,  3-го и 4-го. Средства, доступные после 1-го этапа равны $ x_{1} $. Функциональное уравнение имеет вид:

$$ f_2 ( x_{1} ) = max_{ 0 \le y_2 \le x_{1} }{\left\{ t_2 ( x_{1}, y_2 ) + f_3 ( x_2 ) )\right\}}= max_{ 0 \le y_2 \le x_{1} }{\left\{ 1{,}4 y_{2}^{2} + 2 \left(x_{1} - y_{2}\right)^{2} + 2{,}72 x_{2}^{2} \right\}}.$$

Здесь $ x_2 $ – сумма оставшихся средств после 2-го этапа (на 2-м этапе было израсходовано $ y_{2} $ средств на предприятии П1 и $ x_{1} - y_{2} $ – на предприятии П2), т.е.

$$ x_2 = 0{,}55 y_{2} + 0{,}6 x_{1} - 0{,}6 y_{2} = 0{,}6 x_{1} - 0{,}05 y_{2} .$$

Заменяя $ x_2 $ его выражением через $ x_{1} $ и $ y_{2} $, окончательно получаем функциональное уравнение:

$$ f_2 ( x_{1} ) = max_{ 0 \le y_2 \le x_{1} }{\left\{ 2{,}979 x_{1}^{2} - 4{,}163 x_{1} y_{2} + 3{,}407 y_{2}^{2} \right\}}.$$

Находим значение $ y_{2} $, при котором функция, заключенная в фигурные скобки, на отрезке [0,~$ x_{1} $] достигает наибольшего значения (для простоты обозначим ее через $ Z_2 $). Так как $ x_{1} $ – для 2-го этапа есть величина постоянная, то

$$ Z_2 = Z_2 ( y_2 ) =  2{,}979 x_{1}^{2} - 4{,}163 x_{1} y_{2} + 3{,}407 y_{2}^{2} $$

есть уравнение параболы, ветви которой направлены вверх. Наибольшее значение на отрезке [0, $ x_{1} $] функция $ Z_2 ( y_2 ) $ принимает на одном из его концов. Имеем:

$$ Z_2 ( 0 ) = 2{,}979 x_{1}^{2}, $$

$$ Z_2 ( x_{1} ) = 2{,}223 x_{1}^{2}. $$

Поэтому $ f_2 ( x_{1} ) = 2{,}979 x_{1}^{2} $. Следовательно, максимальный доход на 2-м этапе будет достигнут в том случае, если в начале его все оставшиеся средства вложить в развитие предприятия П2.

4. Этапы 1-й, 2-й, 3-й и 4-й. Функциональное уравнение имеет вид:

$$ f_1 ( x ) = max_{ 0 \le y_1 \le x }{\left\{ t_1 ( x, y_1 ) + f_2 ( x_1 ) )\right\}}= max_{ 0 \le y_1 \le x }{\left\{ 1{,}4 y_{1}^{2} + 2 \left(x - y_{1}\right)^{2} + 2{,}979 x_{1}^{2} \right\}}.$$

Здесь $ x_1 $ – сумма оставшихся средств после 1-го этапа (на 1-м этапе было израсходовано $ y_{1} $ средств на предприятии П1 и $ x - y_{1} $ – на предприятии П2), т.е.

$$ x_1 = 0{,}55 y_{1} + 0{,}6 x - 0{,}6 y_{1} = 0{,}6 x - 0{,}05 y_{1} .$$

Заменяя $ x_1 $ его выражением через $ x $ и $ y_{1} $, окончательно получаем функциональное уравнение:

$$ f_1 ( x ) = max_{ 0 \le y_1 \le x }{\left\{ 3{,}073 x^{2} - 4{,}179 x y_{1} + 3{,}407 y_{1}^{2} \right\}}.$$

Находим значение $ y_{1} $, при котором функция, заключенная в фигурные скобки, на отрезке [0,~$ x $] достигает наибольшего значения (для простоты обозначим ее через $ Z_1 $). Так как $ x $ – для 1-го этапа есть величина постоянная, то

$$ Z_1 = Z_1 ( y_1 ) =  3{,}073 x^{2} - 4{,}179 x y_{1} + 3{,}407 y_{1}^{2} $$

есть уравнение параболы, ветви которой направлены вверх. Наибольшее значение на отрезке [0, $ x $] функция $ Z_1 ( y_1 ) $ принимает на одном из его концов. Имеем:

$$ Z_1 ( 0 ) = 3{,}073 x^{2}, $$

$$ Z_1 ( x ) = 2{,}301 x^{2}. $$

Поэтому $ f_1 ( x ) = 3{,}073 x^{2} $. Следовательно, максимальный доход на 1-м этапе будет достигнут в том случае, если в начале его все выделенные средства вложить в развитие предприятия П2.

\noindent\rule{\textwidth}{1pt}\newline

Найденное оптимальное управление справедливо для любого $ x > 0 $, поэтому, не придавая $ x $ определенного значения, определим величину средств, подлежащих перераспределению на каждом году планируемого периода.

На основании полученного решения можно сделать вывод, что оптимальное управление процессом распределения выделенных средств состоит в следующем:

\begin{itemize}

\item В начале первого года все средства $ x $ вкладывают в предприятие П2, и их количество уменьшается до $ 0{,}6 x $.

\item В начале 2 года остаток средств $ 0{,}6 x $ вкладывают в предприятие П2, и их количество уменьшается до величины $ 0{,}36 x $.

\item В начале 3 года остаток средств $ 0{,}36 x $ вкладывают в предприятие П2, и их количество уменьшается до величины $ 0{,}216 x $.

\item В начале 4 года остаток средств $ 0{,}216 x $ вкладывают в предприятие П2, и их количество уменьшается до величины $ 0{,}13 x $.

\end{itemize}

При таком распределении средств за 4 лет будет получен максимальный доход, равный $ f(x) = 3{,}073 x^{2} $.

\end{document}